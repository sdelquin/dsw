\documentclass{beamer}

\usepackage[utf8]{inputenc}
\usepackage[T1]{fontenc}
\usepackage[spanish]{babel}
\usepackage{amsmath}
\usepackage{pifont}
\usepackage{graphicx}

\setbeamertemplate{navigation symbols}{}

\usetheme{Berlin}
\usefonttheme{professionalfonts}

\newcommand{\link}[2][]{%
  \href{#1}{\textcolor{blue}{\underline{\smash{#2}}}}%
}

\title{Desarrollo web en entorno servidor}
\subtitle{Desarrollo de Aplicaciones Web}
\author{Sergio Delgado Quintero}
\institute{IES Puerto de la Cruz - Telesforo Bravo}
\date{\today}

\begin{document}

\begin{frame}
    \titlepage
\end{frame}

\begin{frame}{Tabla de Contenidos}
    \tableofcontents
\end{frame}

\section{El profe}
\begin{frame}{Algo sobre mi}
    \begin{itemize}
        \item Sergio Delgado Quintero.
        \item Estudié \textbf{Ingeniería Informática} en la \textit{Universidad de La Laguna}.
        \item Catedrático de Enseñanza Secundaria especialidad en Informática.
        \item Casi 20 años programando (y enseñando) \textbf{Python}.
        \item Cofundador de \link[https://pythoncanarias.es]{Python Canarias}.
        \item Muchos proyectos desarrollados.
        \item Contacto: \texttt{sdelqui@gobiernodecanarias.org}
    \end{itemize}
\end{frame}

\section{El módulo}
\begin{frame}{Módulo: Desarrollo web en entorno servidor}
    \begin{itemize}
        \item Segundo curso del CFGS Desarrollo de Aplicaciones Web.
        \item 7 horas semanales.
        \item Lenguaje de programación: \textbf{Python}.
        \item Aula virtual CAMPUS: \link[https://peq.es/dsw]{peq.es/dsw}
        \item Contenidos: \link[https://peq.es/django]{peq.es/django}
        \item Máquina virtual: \textbf{VirtualBox}.
    \end{itemize}
\end{frame}

\section{Sistema de evaluación}
\begin{frame}{Instrumentos de evaluación}
    \begin{itemize}
        \item Pruebas objetivas [PO] (\textbf{60\%})
        \begin{itemize}
            \item Pruebas objetivas teóricas [POT] (\textbf{25\%})
            \item Pruebas objetivas prácticas [POP] (\textbf{75\%})\footnote{Es necesario superar para realizar media.}
        \end{itemize}
        \item Tareas evaluables [TE] (\textbf{40\%})\footnote{Se realizará por grupos.}
        \begin{itemize}
            \item Tareas evaluables: Ejercicios [TEE] (\textbf{50\%})
            \item Tareas evaluables: Proyectos [TEP] (\textbf{50\%})
        \end{itemize}
    \end{itemize}

    \fbox{\tiny POP/TEP \ding{43} $0.7 * \text{tests} + 0.3 * \text{código}$}

    \begin{block}{\tiny{Cálculo de la nota}}
        \vspace{-2em}
        \begin{align*}
        N_{ev} =\ & 0.60 * (0.25 * POT + 0.75 * POP) +\\
                & 0.40 * (0.5 * TEE + 0.5 * TEP)
        \end{align*}
        \vspace{-2em}
    \end{block}

\end{frame}

\begin{frame}{Criterios de calificación}
    \begin{tabular}{|l|l|l|l|}
        \hline
        & 1ª Evaluación & 2ª Evaluación & 3ª Evaluación\\
        \hline
        1ª Evaluación & 100\% & - & -\\
        \hline
        2ª Evaluación & 30\%  & 70\% & -\\
        \hline
    \end{tabular}\\[1em]

    \begin{block}{\tiny{Cálculo de la nota}}
        \vspace{-1.5em}
        \begin{align*}
            N_1 &= N_{ev}\\
            N_2 &= 0.3 * N_1 + 0.7 * N_{ev}
        \end{align*}
    \end{block}
\end{frame}

\begin{frame}{Resultados de aprendizaje}
    \framesubtitle{$[RA1-RA3]$}
    \begin{itemize}
        \item \textbf{RA1}: Selecciona las arquitecturas y tecnologías de programación web en entorno servidor, analizando sus capacidades y características propias.
        \item \textbf{RA2}: Escribe sentencias ejecutables por un servidor web reconociendo y aplicando procedimientos de integración del código en lenguajes de marcas.
        \item \textbf{RA3}: Escribe bloques de sentencias embebidos en lenguajes de marcas, seleccionando y utilizando las estructuras de programación.
    \end{itemize}
\end{frame}

\begin{frame}{Resultados de aprendizaje}
    \framesubtitle{$[RA4-RA6]$}
    \begin{itemize}
        \item \textbf{RA4}: Desarrolla aplicaciones web embebidas en lenguajes de marcas analizando e incorporando funcionalidades según especificaciones.
        \item \textbf{RA5}: Desarrolla aplicaciones web identificando y aplicando mecanismos para separar el código de presentación de la lógica de negocio.
        \item \textbf{RA6}: Desarrolla aplicaciones web de acceso a almacenes de datos, aplicando medidas para mantener la seguridad y la integridad de la información.
    \end{itemize}
\end{frame}

\begin{frame}{Resultados de aprendizaje}
    \framesubtitle{$[RA7-RA9]$}
    \begin{itemize}
        \item \textbf{RA7}: Desarrolla servicios web reutilizables y accesibles mediante protocolos web, verificando su funcionamiento.
        \item \textbf{RA8}: Genera páginas web dinámicas analizando y utilizando tecnologías y frameworks del servidor web que añadan código al lenguaje de marcas.
        \item \textbf{RA9}: Desarrolla aplicaciones web híbridas seleccionando y utilizando tecnologías, frameworks servidor y repositorios heterogéneos de información.
    \end{itemize}
\end{frame}

\section{Unidades de trabajo}

\begin{frame}{Secuenciación}
    \begin{tabular}{|l|l|l|}
        \hline
        \textbf{Unidad} & \textbf{Nombre} & \textbf{Trimestre}\\
        \hline
        UT1 & Introducción a la programación web & \textcolor{purple}{T1}\\
        \hline
        UT2 & Django básico & \textcolor{purple}{T1}\\
        \hline
        UT3 & Django intermedio & \textcolor{purple}{T1}\\
        \hline
        \hline
        UT4 & Django avanzado & \textcolor{teal}{T2}\\
        \hline
        UT5 & Django especializado & \textcolor{teal}{T2}\\
        \hline
    \end{tabular}
\end{frame}

\begin{frame}{Relación UT/RA}
    \begin{tabular}{|l|l|l|l|l|l|l|l|l|}
        \hline
            & UT1  & UT2   & UT3  & UT4   & UT5\\
        \hline
        RA1 & 90\% & 10\%  &      &       &\\
        \hline
        RA2 &      & 100\% &      &       &\\
        \hline
        RA3 &      & 20\%  & 30\% & 40\%  & 10\%\\
        \hline
        RA4 &      &       & 30\% & 40\%  & 30\%\\
        \hline
        RA5 &      &       & 40\% & 40\%  & 20\%\\
        \hline
        RA6 &      & 15\%  & 25\% & 30\%  & 30\%\\
        \hline
        RA7 &      &       &      &       & 100\%\\
        \hline
        RA8 &      &       &      & 100\% &\\
        \hline
        RA9 &      &       &      &       & 100\%\\
        \hline
    \end{tabular}
\end{frame}

\begin{frame}{Entregables}
    \begin{tabular}{|l|l|l|l|l|}
        \hline
            & POT       & POP      & TEE       & TEP\\
        \hline
        UT1 & \ding{51} &          &           &\\
        \hline
        UT2 & \ding{51} &\ding{51} &           & \ding{51}\\
        \hline
        UT3 & \ding{51} &\ding{51} & \ding{51} & \ding{51}\\
        \hline
        UT4 & \ding{51} &\ding{51} & \ding{51} & \ding{51}\\
        \hline
        UT5 & \ding{51} &\ding{51} &           & \ding{51}\\
        \hline
    \end{tabular}
\end{frame}

\section{Otros aspectos}

\begin{frame}{Material}
    \begin{columns}
        \begin{column}[t]{0.5\textwidth}
            \begin{block}{Material obligatorio}
                \begin{itemize}
                    \item Cuaderno.
                    \item Bolígrafo.
                \end{itemize}
            \end{block}
        \end{column}
        \begin{column}[t]{0.5\textwidth}
            \begin{block}{Material recomendado}
                \begin{itemize}
                    \item Disco duro externo USB.
                \end{itemize}
            \end{block}
        \end{column}
    \end{columns}
\end{frame}

\begin{frame}{Horario}
    \begin{tabular}{|l|l|l|l|l|l|}
        \hline
         & Lunes & Martes & Miércoles & Jueves & Viernes\\
        \hline        
        14:30 - 15:25 & \textcolor{purple}{DSW} & \textcolor{purple}{DSW}& & &\textcolor{purple}{DSW}\\
        \hline
        15:25 - 16:20 &\textcolor{purple}{DSW} &\textcolor{purple}{DSW} & & &\textcolor{purple}{DSW}\\
        \hline
        16:20 - 17:15 & & & & &\\
        \hline
        17:15 - 17:45 & {\tiny Recreo} & {\tiny Recreo} & {\tiny Recreo} & {\tiny Recreo} & {\tiny Recreo}\\
        \hline
        17:45 - 18:40 & & &\textcolor{purple}{DSW} & &\\
        \hline
        18:40 - 19:35 & & & & &\\
        \hline
        19:35 - 20:30 & & & & &\\
        \hline
    \end{tabular}
\end{frame}

\begin{frame}{IA}

    {\Huge IA significa Inteligencia Artificial}\\[1em]
    \centering
    \includegraphics[width=0.3\textwidth]{images/robot.png}

\end{frame}

\begin{frame}{Algo sobre ti}
    \begin{enumerate}
        \item ¿Cómo te llamas? ¿Cómo quieres que te llamen?
        \item ¿Qué lenguajes de programación has manejado?
        \item ¿Dónde te ves al acabar el ciclo?
        \item ¿Cuál es tu hobby?
        \item ¿Cuál es tu artista/grupo favorito de música?
    \end{enumerate}
\end{frame}

\end{document}
